\documentclass[final]{beamer}
\usetheme{RJH}
\usepackage[width=180,height=120,scale=1.5,debug]{beamerposter}
\usepackage[absolute,overlay]{textpos}
\setlength{\TPHorizModule}{1cm}
\setlength{\TPVertModule}{1cm}

\usepackage{graphicx}
\usepackage{caption}
\usepackage{subcaption}

\title{Modular Inverse Reinforcement Learning on Human Motion}
\author{Shun Zhang, Matthew Tong, Mary Hayhoe, Dana Ballard\\
Department of Computer Science, Center for Perceptual Systems\\
University of Texas at Austin}
\footer{}

\begin{document}
\begin{frame}{} 

% left
\begin{textblock}{37}(1,8)
\begin{block}{Introduction}
blahh
\end{block}
\end{textblock}

% middle
\begin{textblock}{37}(40,8)
\begin{block}{Multi-objective Domain}
\begin{figure}[h]
\centering
\includegraphics[width=0.2\textwidth]{human.jpg}
\includegraphics[width=0.6\textwidth]{env.png}
\caption{(Left) A human subject with a head mounted display (HMD) and trackers
for the eye, head, and body.  (Right) The environment the human can see through
the HMD.  The red cubes represent obstacles. The blue balls represent targets.
There is also a gray path on the ground that the human subject can follow.}
\label{fig:avatar}
\end{figure}
\end{block}

\begin{block}{Experiments}
\begin{figure}[h]
\centering
\begin{subfigure}[b]{0.4\textwidth}
\includegraphics[width=\textwidth]{task_1.png}
\caption{Path module only,\\$w = (0.039, 0.0, 0.960)$. }
\end{subfigure}
\begin{subfigure}[b]{0.4\textwidth}
\includegraphics[width=\textwidth]{task_2.png}
\caption{Obstacle + Path,\\$w = (0.081, 0.264, 0.654)$. }
\end{subfigure}
\begin{subfigure}[b]{0.4\textwidth}
\includegraphics[width=\textwidth]{task_3.png}
\caption{Target + Path, \\$w = (0.254, 0.089, 0.655)$. }
\end{subfigure}
\begin{subfigure}[b]{0.4\textwidth}
\includegraphics[width=\textwidth]{task_4.png}
\caption{Target + Obstacle + Path, \\$w = (0.215, 0.414, 0.369)$. }
\end{subfigure}
\caption{The trajectories of humans and the agent in the four tasks. Targets are blue and obstacles are red. The
black lines are trajectories of human subjects, and the green lines are
trajectories of the learning agent by using the optimum weights, $w$, derived
from modular inverse reinforcement learning. Weights for each task are given as (target,
obstacle, path).}
\label{fig:exp}
\end{figure}

\end{block}
\end{textblock}

% right
\begin{textblock}{37}(79,8)
\begin{block}{Experiments}
\begin{figure}[h]
\centering
\begin{subfigure}[b]{0.24\textwidth}
\includegraphics[width=\textwidth]{contact1.png}
\caption{Path module only. }
\end{subfigure}
\begin{subfigure}[b]{0.24\textwidth}
\includegraphics[width=\textwidth]{contact2.png}
\caption{Obstacle + Path. }
\end{subfigure}
\begin{subfigure}[b]{0.24\textwidth}
\includegraphics[width=\textwidth]{contact3.png}
\caption{Target + Path. }
\end{subfigure}
\begin{subfigure}[b]{0.24\textwidth}
\includegraphics[width=\textwidth]{contact4.png}
\caption{All modules. }
\end{subfigure}
\caption{Number of targets hit and number of obstacles hit of the human subjects 
and the agent.}
\label{fig:stats}
\end{figure}

\begin{figure}[h]
\centering
\begin{subfigure}[b]{0.24\textwidth}
\includegraphics[width=\textwidth]{objValuesTask1.png}
\caption{Path following.}
\end{subfigure}
\begin{subfigure}[b]{0.24\textwidth}
\includegraphics[width=\textwidth]{objValuesTask2.png}
\caption{Obstacle + Path. }
\end{subfigure}
\begin{subfigure}[b]{0.24\textwidth}
\includegraphics[width=\textwidth]{objValuesTask3.png}
\caption{Target + Path. }
\end{subfigure}
\begin{subfigure}[b]{0.24\textwidth}
\includegraphics[width=\textwidth]{objValuesTask4.png}
\caption{All modules. }
\end{subfigure}
\caption{Heatmaps of the $\log$ of the values of Equation~\ref{eq:irl} for
different weights for the four tasks, respectively. The white zones indicate
higher probabilities. The weights of all three modules sum to 1, so we only show
the weights on the target and the obstacle modules.
}
\label{fig:heatmap}
\end{figure}
\end{block}
\end{textblock}

\end{frame}
\end{document}
