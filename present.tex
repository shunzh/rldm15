\documentclass[final]{beamer}
\usetheme{RJH}
\usepackage[width=180,height=120,scale=1.5,debug]{beamerposter}
\usepackage[absolute,overlay]{textpos}
\setlength{\TPHorizModule}{1cm}
\setlength{\TPVertModule}{1cm}

\usepackage{graphicx}
\usepackage{caption}
\usepackage{subcaption}

\title{Modular Inverse Reinforcement Learning on Human Motion}
\author{Shun Zhang, Matthew Tong, Mary Hayhoe, Dana Ballard\\
Department of Computer Science, Center for Perceptual Systems\\
University of Texas at Austin}
\footer{}

\begin{document}
\begin{frame}{} 

% left
\begin{textblock}{35}(1,8)
\begin{block}{Introduction}
\begin{itemize}
\item
Humans are able to learn and carry out very complex tasks involving 
many different objectives, while most reinforcement learning algorithms suffer
from the curse of dimensionality.
\item
One promising possibility is that the complex task can be broken down into 
{\bf sub-tasks} that are each learned separately.
\item
This {\bf decomposition} allows the behavior in the complex task to be chosen based on
the value of a weighted sum of individual sub-tasks.
\item
Our experimental analysis shows that the modular reinforcement can be a 
good model of predicting {\bf human subjects'} sub-task priorities in a way that
explains their behavioral choices.
\item
We use the {\bf Modular Inverse Reinforcement Learning} approach to analyze
human subjects' behaviors.
\end{itemize}
\end{block}

\begin{block}{Modular Inverse Reinforcement Learning}
\begin{itemize}
\item {\bf Factored MDP.}
Define $S = S_1 \times S_2 \times \cdots \times S_m$, where $S_i$ is a state
component. We expect the state components are less correlated. The transition
function can be defined as a Dynamic Bayesian Network between two time frames of
state components.
\item {\bf Q-Decomposition.}
We further decompose the Q function.
Assume that the global Q function is a weighted sum of all $Q_i$, where $Q_i$ is
the Q function for $S_i$, the $i$-th state component.
$$Q(s, a) = \sum_i w_i Q_i (s_i, a)$$
where $w_i$ is a weight scalar. $w_i \geq 0, \sum_i w_i = 1$.
Now the original MDP is {\bf modularized} into sub-MDPs.
$$MDP_i = <S_i, A, T, R_i, \gamma>$$
\item {\bf Modular Inverse Reinforcement Learning.}
The objective is to recover the weights given observed samples, $(s^{(i)},
a^{(i)})$, and sub-MDPs.  We maximize the likelihood of observing such samples,
$$\max_w \prod_t \frac{e^{\eta Q(s^{(t)}, a^{(t)})}}{\sum_b e^{\eta Q(s^{(t)},
b)}}$$
where $s^{(t)}$ is the state at time $t$, and $a^{(t)}$ is the action at time
$t$, which are both from samples. $Q(s, a) = \sum_i w_i Q_i(s_i, a)$, as defined
before. $\eta$ is a hyperparameter that determines the consistency of human's
behavior.
\end{itemize}
\end{block}
\end{textblock}

% middle
\begin{textblock}{37}(38,8)
\begin{block}{Multi-objective Domain}
\begin{figure}[h]
\centering
\includegraphics[width=0.2\textwidth]{human.jpg}
\includegraphics[width=0.6\textwidth]{env.png}
\caption{(Left) A human subject with a head mounted display (HMD) and trackers
for the eye, head, and body.  (Right) The environment the human can see through
the HMD.  The red cubes represent obstacles. The blue balls represent targets.
There is also a gray path on the ground that the human subject can follow.}
\label{fig:avatar}
\end{figure}
\end{block}

\begin{block}{Experiments}
\begin{figure}[h]
\centering
\begin{subfigure}[b]{0.4\textwidth}
\includegraphics[width=\textwidth]{task_1.png}
\caption{Path module only,\\$w = (0.039, 0.0, 0.960)$. }
\end{subfigure}
\begin{subfigure}[b]{0.4\textwidth}
\includegraphics[width=\textwidth]{task_2.png}
\caption{Obstacle + Path,\\$w = (0.081, 0.264, 0.654)$. }
\end{subfigure}
\begin{subfigure}[b]{0.4\textwidth}
\includegraphics[width=\textwidth]{task_3.png}
\caption{Target + Path, \\$w = (0.254, 0.089, 0.655)$. }
\end{subfigure}
\begin{subfigure}[b]{0.4\textwidth}
\includegraphics[width=\textwidth]{task_4.png}
\caption{Target + Obstacle + Path, \\$w = (0.215, 0.414, 0.369)$. }
\end{subfigure}
\caption{The trajectories of humans and the agent in the four tasks. Targets are blue and obstacles are red. The
black lines are trajectories of human subjects, and the green lines are
trajectories of the learning agent by using the optimum weights, $w$, derived
from modular inverse reinforcement learning. Weights for each task are given as (target,
obstacle, path).}
\label{fig:exp}
\end{figure}

\end{block}
\end{textblock}

% right
\begin{textblock}{39}(77,8)
\begin{block}{Experiments}
\begin{figure}[h]
\centering
\begin{subfigure}[b]{0.24\textwidth}
\includegraphics[width=\textwidth]{contact1.png}
\caption{Path module only.}
\end{subfigure}
\begin{subfigure}[b]{0.24\textwidth}
\includegraphics[width=\textwidth]{contact2.png}
\caption{Obstacle + Path.}
\end{subfigure}
\begin{subfigure}[b]{0.24\textwidth}
\includegraphics[width=\textwidth]{contact3.png}
\caption{Target + Path.}
\end{subfigure}
\begin{subfigure}[b]{0.24\textwidth}
\includegraphics[width=\textwidth]{contact4.png}
\caption{All modules.}
\end{subfigure}
\caption{Number of targets hit and number of obstacles hit of the human subjects 
and the agent.}
\label{fig:stats}
\end{figure}

\begin{figure}[h]
\centering
\begin{subfigure}[b]{0.24\textwidth}
\includegraphics[width=\textwidth]{objValuesTask1.png}
\caption{Path module only.}
\end{subfigure}
\begin{subfigure}[b]{0.24\textwidth}
\includegraphics[width=\textwidth]{objValuesTask2.png}
\caption{Obstacle + Path. }
\end{subfigure}
\begin{subfigure}[b]{0.24\textwidth}
\includegraphics[width=\textwidth]{objValuesTask3.png}
\caption{Target + Path. }
\end{subfigure}
\begin{subfigure}[b]{0.24\textwidth}
\includegraphics[width=\textwidth]{objValuesTask4.png}
\caption{All modules. }
\end{subfigure}
\caption{Heatmaps of the $\log$ of the likelihood for
different weights for the four tasks, respectively. The white zones indicate
higher probabilities. The vertical axis is the weight of the target module. The
horizontal axis is the weight of the obstacle module.
}
\label{fig:heatmap}
\end{figure}
\end{block}

\begin{block}{Conclusion}
\begin{itemize}
\item
We analyzed human behavior using inverse modular reinforcement learning.
\item
The experimental results show that modular reinforcement learning can explain
human behavior well, even though the performance of the agent is currently
inferior to human subjects'.
\end{itemize}
\end{block}

\begin{block}{Following Work}
\begin{itemize}
\item
Learning weights (or rewards) and discounters of sub-MDPs simultaneously.
\item
Testing in gridworld domains, with hundreds of sub-MDPs. We compared Modular IRL
with Bayesian IRL in this condition.
\item
Evaluation by angular differences in policies, and likelihood of trajectories.
This is compared with other baseline agents.
\end{itemize}

{\bf More details in our paper to appear in a future conference!}
\end{block}

\begin{block}{Acknowledgment}
\begin{itemize}
\item Funding?
\end{itemize}
\end{block}

\end{textblock}

\end{frame}
\end{document}
